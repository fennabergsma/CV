\documentclass[12pt]{article}

\usepackage{geometry}
 \geometry{
 a4paper,
 total={170mm,257mm},
 left=5mm
 }

\usepackage{fenna-files/packages}
\usepackage{fenna-files/commands}

%Setup hyperref package, and colours for links
\usepackage{hyperref}
\definecolor{linkcolour}{rgb}{0,0.2,0.6}
\hypersetup{colorlinks,breaklinks,urlcolor=linkcolour, linkcolor=linkcolour}

\pagestyle{empty}

\begin{document}
\phantom{x}
\bigskip
\bigskip

\begin{tabular}{p{3cm}p{13cm}}
& \Huge \sffamily{\tsc{Fenna Bergsma}} \\&\\
\end{tabular}

\bigskip
\bigskip
\bigskip

\renewcommand{\arraystretch}{1.05}
\begin{tabular}{p{3cm}rl}
& \multicolumn{1}{l}{\Large{\sffamily{Personal}}} & \\
	& & \\
    & {Born:} & August 26, 1991\\
		& & Boarnsterhim, The Netherlands\\
    & {Nationality:} & Dutch\\&\\&\\
\end{tabular}

\begin{tabular}{p{3cm}rl}
& \multicolumn{1}{l}{\Large{\sffamily{Contact}}} & \\
& & \\
& {Address:}
 & Goethe-Universität\\
& & Institut für Linguistik\\
& & Norbert-Wollheim-Platz 1\\
& & 60629 Frankfurt am Main\\
& & Germany\\
  &  {E-mail:}     & \href{mailto:bergsma@em.uni-frankfurt.de}{bergsma@em.uni-frankfurt.de}\\
  &  {Website:}
	& \href{http://user.uni-frankfurt.de/bergsma/}{user.uni-frankfurt.de/bergsma/} \\&\\&\\&\\
\end{tabular}

\renewcommand{\arraystretch}{1.15}
\begin{tabular}{p{3cm}p{13cm}}
& \multicolumn{1}{l}{\Large{\sffamily{Education}}} \\
& \\
\sffamily10/2017-09/2020 & PhD \textsc{Linguistics}, {Goethe-Universität}, Frankfurt\\
& Research Training Group on Nominal Modification\\
& Preliminary dissertation title: \tit{Case competition in headless relatives} \\
\sffamily09/2019, 02/2019-04/2019 & research stay at {Masarykova univerzita}, Brno\\
\sffamily03/2018-05/2018 & research stay at {University of Pennsylvania}, Philadelphia\\
\sffamily08/2018 & EGG summer school 2018, {University of Banja Luka}\\
\sffamily08/2017 & EGG summer school 2017, {Univerzita Palackého v Olomouci}\\&\\
\sffamily09/2013-08/2015 & Research Master \textsc{Linguistics}, {Universiteit van Amsterdam}\\
& Thesis: \emph{To serve double duty under syncretism - How Nanosyntax and grafting account for the free relative construction}\\&\\
\sffamily09/2010-08/2013 & Bachelor \textsc{Linguistics}, {Rijksuniversiteit Groningen} - \small\emph{cum laude}\\
& Thesis: \emph{Does Knowledge about Phonological Correspondences Contribute to the Intelligibility of a Related Language? A Study with Speakers of Dutch Learning Frisian.}\\
\sffamily09/2010-08/2013& Honours College program (25\% extra credits) \\
\end{tabular}

\renewcommand{\arraystretch}{1.25}
\begin{tabular}{p{3cm}p{13cm}}
  & \multicolumn{1}{l}{\Large{\sffamily{Publications}}} \\
  & \\
\sffamily under review & Bergsma, Fenna. The instrumental \tsc{r}-pronoun and postposition in Dutch. In: E. Anagnostopoulou and C. Sevdali. \textit{On the place of case in grammar.} \\
\sffamily2019 & Bergsma, Fenna. Mismatches in free relatives - grafting nanosyntactic trees. \emph{Glossa: A Journal of General Linguistics 4}(1), 119. DOI: \href{http://doi.org/10.5334/gjgl.821}{http://doi.org/10.5334/gjgl.821}\\
\sffamily2019 & Bergsma, Fenna. \href{https://repository.upenn.edu/pwpl/vol25/iss1/6}{The Role of Prepositions in Case Mismatches in Free Relatives.} In: A. Creemers and C. Richter. \textit{University of Pennsylvania Working Papers in Linguistics 25.1.} 41-50. \\
\sffamily2014 & Bergsma, Fenna, Femke Swarte and Charlotte Gooskens. Does Instruction about Phonological Correspondences Contribute to the Intelligibility of a Related Language? A Study with Speakers of Dutch Learning Frisian. \emph{Dutch Journal of Applied Linguistics, 3}(1), 45-61. DOI: \href{https://doi.org/10.1075/dujal.3.1.03ber}{https://doi.org/10.1075/dujal.3.1.03ber}\\&\\&\\
\end{tabular}


\begin{tabular}{p{3cm}p{13cm}}
& \multicolumn{1}{l}{\Large{\sffamily{Talks}}} \\
& \\
& \multicolumn{1}{l}{\large{\tsc{\sffamily{Invited talks}}}} \\
\sffamily2018 & ``\emph{Waarmee} and \emph{met wat} in Dutch free relatives'', \emph{Oberseminar English Linguistics (Syntax-Semantics)}, Georg-August-Universität Göttingen, October 30.\\
\sffamily2018 & ``Grafting nanosyntactic trees: an analysis of case mismatches in free relatives'', \emph{Kolloquium Topics in syntax and its interfaces}, Universität Leipzig,  June 26.\\&\\
& \multicolumn{1}{l}{\large{\tsc{\sffamily{Conference talks (peer-reviewed)}}}} \\
\sffamily2019 & ``Gender: a Matter of Size – On Individuation, Mass, and Diminutives in Dutch'', \textit{SinFonIJA 12}, Masarykova univerzita, September 12 - 14. With Jan Don\\
\sffamily2019 & ``Verum focus in Frisian'', poster at \emph{GLOW 42}, Universitetet i Oslo, May 7-11.\\
\sffamily2019 & ``PPs and DPs in non-matching free relatives'', \emph{Exploring Nanosyntax, LSA annual meeting}, New York, January 3-6.\\
\sffamily2018 & ``PPs and DPs in free relatives: \emph{waarmee} and \emph{met wat} in Dutch'', \emph{On the place of case in grammar (PlaCiG)}, Rethymnon, October 18-20.\\
\sffamily2018 & ``Case mismatches in free relative constructions'', \emph{CGG 28}, Universitat Rovira i Virgili, May 30-June 1.\\
\sffamily2018 & ``Mismatches in free relatives'', \emph{GLOW 41}, Research Institute for Linguistics of the Hungarian Academy of Sciences, April 10-14.\\
\sffamily2018 & ``The power of syncretisms: how syncretisms can serve double duty'', \emph{Penn Linguistics Conference 42}, University of Pennsylvania, March 23-25.\\
\sffamily2018 & ``Syncretism = shared syntax + shared spellout'' \emph{ConSOLE XXVI}, University College London, February 14-16.\\
\end{tabular}

\begin{tabular}{p{3cm}p{13cm}}
\sffamily2016 & ``To serve double duty under syncretism'' \emph{ConSOLE XXIV}, University of York and York St John University, January 6-8. With Jan Don\\
\sffamily2015 & ``To serve double duty under syncretism'' \emph{Morphologydays 2015}, Katholieke Universiteit Leuven, December 17-18. With Jan Don\\
\sffamily2013 & ``Does Knowledge About Linguistic Differences Contribute to Receptive Multilingualism? - A Pilot Study with Speakers of Dutch Learning Frisian'' \emph{ExAPP 2013: Experimental Approaches to Perception and Production of Language Variation}, Københavns Universitet, March 20.\\&\\
& \multicolumn{1}{l}{\large{\sffamily{\tsc{Other talks (non peer-reviewed)}}}} \\
\sffamily2019 & ``Towards deriving a typology of case mismatches in free relatives'', \emph{Colloquium Graduiertenkolleg Nominal Modification}, October 22.\\
\sffamily2018 & ``Mismatches in free relatives - grafting nanosyntactic trees'', \emph{Nanosyntax Weblab}, online, November 30.\\
2018 & ``On the distribution of \emph{waarmee} and \emph{met wat} in Dutch free relatives'', \emph{Colloquium Graduiertenkolleg Nominal Modification}, Goethe-Universität Frankfurt, October 16.\\
\sffamily2018 & ``Mismatches in free relatives'', \emph{GK Nominal Modification 2018 Summer Retreat}, Fulda, May 25-26.\\
\sffamily2017 & ``Syncretism = shared syntax + shared spellout'' \emph{Colloquium Graduiertenkolleg Nominal Modification}, Goethe-Universität Frankfurt, December 5.\\
\sffamily2015 & ``(De) Mij(n(e(s))) - Complex Possessives in Dutch'' \emph{Taalkunde in Nederland-dag 2015}, Universiteit Utrecht, February 7. Presented by Yvonne van Baal and Fenna Bergsma \\& With Yvonne van Baal and Jan Don \\
\sffamily2013 & ``How does reality influence performance on false belief tasks?'' \emph{34th TABU Dag 2013}, Rijksuniversiteit Groningen, June 14. Presented by Jidde Jacobi and Maike Tromp. With Jidde Jacobi, Maike Tromp, Maret Hans and Bart Hollebrandse\\
\sffamily2013 & ``Does Knowledge About Linguistic Differences Contribute to Receptive Multilingualism? - A Pilot Study with Speakers of Dutch Learning Frisian'' \emph{Anéla/Viot Juniorendag 2013}, Rijksuniversiteit Groningen, March 3.\\
\sffamily2012 & ``Interventiemethoden voor receptieve meertaligheid - Een case study met Nederlanders die Fries leren verstaan.'' \emph{Dei fan de Fryske Taalkunde 2012}, Fryske Akademy, December 14.\\&\\&\\
\end{tabular}

\begin{tabular}{p{3cm}p{13cm}}
& \multicolumn{1}{l}{\Large{\sffamily{Teaching experience}}} \\
  & \\
\sffamily04/2019-07/2019 & \tit{DP Morphology}, MA Seminar, Summer term.\\
	& Goethe-Universität, Frankfurt\\
\sffamily 02/2012-04/2012 & {Teaching assistant} SPSS Statistics practical\\
	& Rijksuniversiteit Groningen\\
\end{tabular}


\begin{tabular}{p{3cm}p{13cm}}
& \multicolumn{1}{l}{\Large{\sffamily{Organization}}} \\
& \\
\sffamily2019 & Co-organizer, Workshop on formal and experimental approaches to adjectival modification, January 31- February 1, Goethe-Universität, Frankfurt\\
\sffamily2019 & Co-organizer, 45th conference Generative Grammar in the South, July 19-21, Goethe-Universität, Frankfurt\\&\\&\\
\end{tabular}


\begin{tabular}{p{3cm}p{13cm}}
& \multicolumn{1}{l}{\Large{\sffamily{Relevant non-academic work experience}}} \\
& \\
\sffamily09/2015-09/2016, 03/2017-10/2017 & {Project manager} at {Stichting Praktijkleren}, Amersfoort\\
\sffamily09/2006-09/2015 & {Developer of teaching materials} at {Stichting Praktijkleren}, Amersfoort\\&\\&\\
\end{tabular}

\begin{tabular}{p{3cm}p{5cm}p{8cm}}
  & \multicolumn{1}{l}{\Large{\sffamily{Languages}}} & \\
  & &\\
&{native}: &West Frisian, Dutch\\
&{fluent}: &English\\
&{good command}: &German\\&\\&\\
\end{tabular}

\end{document}
