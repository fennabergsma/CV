\documentclass[12pt]{article}

\usepackage[margin=1.15in]{geometry}

\usepackage{../fenna-files/packages}
\usepackage{../fenna-files/commands}

\pagestyle{empty}
\setlength\parindent{0pt}

\usepackage{hyperref}
\definecolor{linkcolour}{rgb}{0,0.2,0.6}
\hypersetup{colorlinks,breaklinks,urlcolor=linkcolour, linkcolor=linkcolour}


\begin{document}
\raggedright

Dear members of the selection committee,\\

\phantom{x}\\

I write this letter to express my interest in the assistant professor position at the Department of General Linguistics at the Albert-Ludwigs-Universität Freiburg. My name is Fenna Bergsma, and I am in the final stages of my PhD within the Research Training Group `Nominal Modification' in Frankfurt. I will hand in my dissertation at the end of July 2020. The PhD position I currently have is limited to three years, and I am encouraged to mainly focus on research. Accordingly, I presented my work at international conferences (a.o. GLOW), and I published an article in a peer-reviewed journal in the second year of my PhD. At the same time, I did not have much opportunity to teach, and gain experience in university management tasks. However, the one MA class I taught convinced me that teaching fits me very well. Moreover, I believe I can transfer management experience that I gained in different places of work (e.g. the industry job I had before I started a PhD). In this letter I elaborate on my (past and future) research projects and on the teaching experience and management skills I acquired over the years.\\

\phantom{x}\\

In my own research I focus mostly on morphosyntactic phenomena from a theoretical generative perspective. I am currently working on three projects within that field: case attraction in headless relatives across languages, \tsc{r}-pronouns and postpositions in Dutch, and connecting gender to the mass/count distinction in Dutch. I discuss these three briefly.\\

\phantom{x}\\

For my dissertation I am working the well-studied phenomenon of case attraction in Germanic headless relatives. I address two main points. First, a language such as Gothic allows for case requirements from the main and relative clause to differ. The only restriction is that the relative pronoun appears in the most complex required case, following the scale \tsc{nom} < \tsc{acc} < \tsc{dat}. The second point I address concerns the cross-linguistic differences that are found. Gothic allows for case attraction in both directions: the case required in the main clause can win over the case required in the relative clause, but also the other way around. Old High German and Modern German do not, they only allow one of the two directions (Old High German only lets the main clause case surface, and German only the relative clause one). My proposals connects the observations to language-internal properties. Both points are a reflex from the language’s morphology. My main goal in the dissertation is to argue that case attraction in headless relatives is not a special property of a small set of languages. Instead, its existence is expected to appear because of how language is organized.\\

\phantom{x}\\

In another morphosyntax project I have been working on the \tsc{r}-pronoun and postposition \tit{waar-mee} `with what’ in Dutch. I argue that this form surfaces when all relevant features form a proper constituent. When this requirement is not met, \tit{met wat} `with what’ appears, realizing the same set of features as \tit{waar-mee} `with what'. This alternation is analyzed as result of regular spellout mechanisms in nanosyntax. A finer decomposition offers an account for three observations: \tsc{r}-pronouns are syncretic with locatives, \tsc{r}-pronouns combine with postpositions and regular pronouns with prepositions, and the instrumental preposition differs phonologically from the instrumental postposition (\tit{met vs}.\tit{ mee}). I am working on also incorporating verbal particles (that pattern in form with postpositions, even though they precede the verb), and extending it to other adpositions in Dutch. The next step is to compare these to German and (my native) Frisian.\\

\phantom{x}\\

The third research topic I am currently working on (in collaboration with Dr. Jan Don from the Universiteit van Amsterdam) is gender in Dutch. We start from the observation that Dutch has some nouns that refer to masses if they combine with the neuter gender determiner, and they refer to counts if they combine with the common gender determiner. Another observation is that the diminutive suffix makes all mass nouns countable, and that the noun plus diminutive always combines with the neuter gendered determiner. At the same time, we also have to allow for randomness: neuter gendered nouns can be mass or count, and common gendered nouns can be mass or count too. We propose an account in which the nouns differ in how they are stored in different sizes in the lexicon, meaning that they spell out more or less features. Combining them with the features that can be realized by the diminutive suffix and the different determiners derives the properties of the noun.\\

\phantom{x}\\

However, my linguistic knowledge and experience is not limited to theoretical morphosyntax. Even before my PhD I already gained quite some research experience during my research-oriented BA and MA programs. I developed and conducted my own experiments, and I worked with child data and statistical models. I highlight five of the projects I conducted.\\

\phantom{x}\\

In a research project in my MA, I investigated the nature of constraints for root infinitives. I examined whether the constraint to eventive verbs in root infinitives of children acquiring their first language might be a result of a lack of epistemic modality. I investigated Dutch root infinitives in second language acquisition from CHILDES using CLAN. In another project in my MA, I researched complementizer agreement in Frisian. I examined the distribution of \emph{-st} and \emph{-sto} as complementizer agreement forms in West-Frisian in emphatic and non-emphatic contexts. I gathered data from a translation task and a felicity judgment task I made myself. A third project in my MA focused on the underlying skills of writing and reading abilities. I explored which underlying skills can account for the correlation between reading and writing abilities. I modeled this using structural equation modeling with the Lavaan package in R.\\

\phantom{x}\\

For my bachelor thesis I designed and set up an experiment. I transcribed the 1,000 most frequent words in Frisian and their translations into Dutch to find the most frequent phonological correspondences between Dutch and Frisian. I developed an intervention to teach the phonological correspondences and two intelligibility tests (one on text and one on word level) as pre- and posttest. I taught the phonological correspondences to Dutch school children. It did not improve the intelligibility of Frisian, but their attitude towards Frisian became more positive. The thesis was a part of a research project on Mutual intelligibility of closely related languages. In a research project in my BA, I worked on Theory of Mind tests. Our small group developed and used our own modified version of the Sally-Anne test, in which the location of the object is unknown after doing a shell game.\\

\phantom{x}\\

Although my main research focus is on theoretical morphosyntax these days, my interest and training extends to numerous subfields of linguistics. I had five years of training in General Linguistics (three in my BA, two in my MA). My bachelor at the Rijksuniversiteit Groningen put a large emphasis on neurolinguistics and experimental linguistics, for example focusing on aphasia and specific language impairment and conducting experiments on child language acquisition. I broadened my perspective more trough courses on language change, speech analysis and eye-tracking during my master at the Universiteit van Amsterdam. I attached lists of courses I took in my BA and MA. I am confident that I can teach many of these subfield.\\

\phantom{x}\\

In my time in Frankfurt, I taught an MA seminar on DP morphology. Besides teaching the students the content of the class, I had another (in my opinion, evenly important) goal. I wanted to help them train skills that are useful inside (but also outside) linguistics. These skills consist, for example, of learning to form and having the courage to ask questions in front of a group, to evaluate arguments critically, and to see the pros and cons of different approaches. At the end of the seminar, I had a group of about ten (mostly young female) students that lively discussed the topics.
\\

\phantom{x}\\

My management experience I gained mostly in the job I had before I started my PhD. I was working as a project manager in a company that develops teaching materials for vocational education. My task was to form sets of assignments based on nation-wide regulations and requests from schools. I worked with several teachers, who were content-specialist on the topic, and who wrote the content of the assignments. Throughout the year, I became increasingly better in providing the teachers with feedback that they would happily implement. After a break for a five-month kayaking trip, the same company happily invited me back to help out wherever needed. One of these tasks was to reorganize the work portfolio of a colleague who fell out with a burn-out. This was a challenging but satisfying job in which I needed to sometimes restart projects content-wise, while keeping work relationships among colleagues intact. I believe these skills will be big help for management tasks within a university.\\

\phantom{x}\\

I hope to have given a good overview of how I could be an asset for the Albert-Ludwigs-Universität Freiburg. Thank you in advance for you consideration. I would be more than happy to answer any further questions.\\

\phantom{x}\\



Sincerely,\\
Fenna Bergsma










\newpage

In the remainder of this document, please find:

\begin{itemize}
  \item My curriculum vitae including a complete list of publications
  \item Copies of three representative writings:
  \begin{itemize}
    \item A recurring pattern (a chapter from my dissertation)
    \item The \tsc{r}-pronoun and postposition \tit{waar-mee} in Dutch (a recent article I wrote)
    \item Mismatches in free relatives - grafting nanosyntactic trees (an article I published in Glossa last year)
  \end{itemize}
\end{itemize}




\end{document}
