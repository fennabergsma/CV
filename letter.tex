\documentclass[12pt]{article}

\usepackage[margin=1.1in]{geometry}

\usepackage{fenna-files/packages}
\usepackage{fenna-files/commands}

\pagestyle{empty}
\setlength\parindent{0pt}

\usepackage{hyperref}
\definecolor{linkcolour}{rgb}{0,0.2,0.6}
\hypersetup{colorlinks,breaklinks,urlcolor=linkcolour, linkcolor=linkcolour}


\begin{document}
% \raggedright

Dear members of the LeibnizDream selection committee,\\

I write this letter to express my interest in a post-doctoral position within the LeibnizDream project. My name is Fenna Bergsma, and I am a morphologist in the final stages of my PhD within the Research Training Group `Nominal Modification' in Frankfurt. In this letter I, first, explain what drives me as a linguist. Second, I elaborate on my experience as a researcher. Third, I discuss the contributions I vision for myself within the project.\\

When I started studying linguistics I wanted to learn how language works in our brains. Within a few years I realized that we know very little about the brain, especially the higher cognitive functions, and I turned to theoretical linguistics. My aim was to contribute to developing a theory of human language. At the same time, neurologists could determine how our brains work. Some day, the brain model could then be mapped onto our language theory. So, what I want to do today is to contribute to this theory of language. Whatever this theory of language is, it is not necessarily supposed to do what the brain does, but it should be a good enough theory within itself. For instance, cross-linguistic differences should be found in a single place, and not in different parts of the model. Therefore, language-specific rules and construction-specific analysis should not exist. The theory that I work in that does that is nanosyntax. Everything except for one variable is kept constant across languages. The functional sequence and feature inventory is the same for all languages, and the spellout algorithm is the same. The only difference between languages is the post-syntactic lexicon. This lexicon consists of syntactic trees that are connected to a phonological and a semantic exponent, which can be mapped onto syntactic structures. This means that the work consists of carefully investigating data, finding out which meaning is expressed by a single morpheme, and comparing this across languages. Looking at child language in which there is a more one-to-one match between language and thought fits very well within that picture.\\

For my dissertation I am working the well-studied phenomenon of case attraction in Germanic headless relatives. I address two main points. First, a language such as Gothic allows for case requirements from the main and relative clause to differ. The only restriction is that the relative pronoun appears in the most complex required case, following the scale \tsc{nom} < \tsc{acc} < \tsc{dat}. I derive that by following work that has established that case is morphologically complex: [\tsc{dat}[\tsc{acc}[\tsc{nom}]]]. I analyze the case attraction phenomena as a result of properties of morphology. Furthermore, I follow the general assumption that ellipsis takes place under identity targets phrases. What follows is that [\tsc{dat}[\tsc{acc}[\tsc{nom}]]] elides [\tsc{acc}[\tsc{nom}]], providing the surface phenomenon that the most complex case surfaces. The second point I address concerns the cross-linguistic differences that are found. Gothic allows for case attraction in both directions: the case required in the main clause can win over the case required in the relative clause, but also the other way around. Old High German and Modern German do not, they only allow one of the two directions (Old High German only lets the main clause case surface, and German only the relative clause one). I am working out a proposal in which this is a reflex from the language’s morphology (again). The relevant correlate in morphology is the shape of the ‘base’ or the relative pronoun: Old High German uses a \tsc{d}-element, Modern German uses a \tsc{wh}-element and Gothic uses a \tsc{d}-element plus an independent complementizer. My main goal in the dissertation is to argue that case attraction in headless relatives is not a special property of a small set of languages. Instead, its existence is expected to appear because of how language is organized.\\

In another recent project I have been working on the \tsc{r}-pronoun and postposition \tit{waar-mee} `with what’ in Dutch. I argue that this form surfaces when all relevant features form a proper constituent. When this requirement is not met, \tit{met wat} `with what’ appears, realizing the same set of features as \tit{waar-mee} `with what'. This alternation is analyzed as result of regular spellout mechanisms in nanosyntax. A finer decomposition offers an account for three observations: \tsc{r}-pronouns are syncretic with locatives, \tsc{r}-pronouns combine with postpositions and regular pronouns with prepositions, and the instrumental preposition differs phonologically from the instrumental postposition (\tit{met vs}.\tit{ mee}). I am working on also incorporating verbal particles (that pattern in form with postpositions, even though the precede the verb), and extending it to other adpositions in Dutch. The next step is to compare these to German and Frisian.\\

Another topic I have been working on (in collaboration with Dr. Jan Don from the University of Amsterdam) is gender in Dutch. We start from the observation that Dutch has some nouns that refer to masses if they combine with the neuter gender determiner, and they refer to counts if they combine with the common gender determiner. Another observation is that the diminutive suffix makes all mass nouns countable, and that the noun plus diminutive always combines with the neuter gendered determiner. At the same time, we also have to allow for randomness: neuter gendered nouns can be mass or count, and common gendered nouns can be mass or count too. We propose an account in which the nouns differ in how they are stored in different sizes in the lexicon, meaning that they spell out more or less features. Combining them with the features that can be realized by the diminutive suffix and the different determiners derives the properties of the noun.\\

In my years in Groningen and Amsterdam I have gained some experience working experimentally and on acquisition. In one research project, I examined whether the constraint to eventive verbs in root infinitives of children acquiring their first language might be a result of a lack of epistemic modality. I investigated Dutch root infinitives in second language acquisition from CHILDES using CLAN. For my bachelor thesis I designed and set up an experiment. I transcribed the 1,000 most frequent words in Frisian and their translations into Dutch to find the most frequent phonological correspondences between Dutch and Frisian. I developed an intervention to teach the phonological correspondences and two intelligibility tests (one on text and one on word level) as pre- and posttest. I taught the phonological correspondences to Dutch school children. It did not improve the intelligibility of Frisian, but their attitude towards Frisian became more positive. The thesis was a part of a research project on Mutual intelligibility of closely related languages.\\

Within the LeibnizDream project I can see my main contributions in developing the morphological theory, which is built on semantic primitives, and translating this to testable hypotheses. What speaks to me about this project is that it aims to develop a system from beginning to end, incorporating all disciplines. In each step of the process, there are experts with a different background and a different main focus. That way, nothing can get lost between the interfaces. Within our Research Training Group I have some experience with being part of group that works in different disciplines but on the same topic. The regular interactions that I have here make me spell out my own ideas and assumptions in more detail, and help me find problems. On the other hand, work from others helps me get a better insight about approaches from different perspectives. In Frankfurt we all have our own dissertations to finish. My hope for the LeibnizDream project is that collaborations will be even more intense, and with that the positive effects of them as well.\\

Currently I am writing up my dissertation, which I will hand in before October 12. That means that January 1 would be good time for me to start a new project.\\

Sincerely,\\
Fenna Bergsma










\newpage

In the remainder of this document, please find:

\begin{itemize}
  \item My curriculum vitae including a complete list of publications
  \item Copies of three representative writings:
  \begin{itemize}
    \item The \tsc{r}-pronoun and postposition \tit{waar-mee} in Dutch (a recent article I wrote)
    \item A recurring pattern (a chapter from my dissertation)
    \item Mismatches in free relatives - grafting nanosyntactic trees (an article I published in Glossa last year)
  \end{itemize}
\end{itemize}




\end{document}
