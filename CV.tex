\documentclass[12pt]{article}

\usepackage{geometry}
 \geometry{
 a4paper,
 total={170mm,257mm},
 left=5mm
 }

\usepackage{fenna-files/packages}
\usepackage{fenna-files/commands}

%Setup hyperref package, and colours for links
\usepackage{hyperref}
\definecolor{linkcolour}{rgb}{0,0.2,0.6}
\hypersetup{colorlinks,breaklinks,urlcolor=linkcolour, linkcolor=linkcolour}

\pagestyle{empty}

\begin{document}
\phantom{x}
\bigskip
\bigskip

\begin{tabular}{p{3cm}l}
& \Huge \sffamily{\tsc{Fenna Bergsma}} \\&\\
\end{tabular}

\bigskip
\bigskip
\bigskip

\begin{tabular}{p{3cm}rl}
& \multicolumn{1}{l}{\Large{\sffamily{Personal}}} & \\
	& & \\
    & {Born:} & 26 August 1991\\
		& & Boarnsterhim, The Netherlands\\
    & {Nationality:} & Dutch\\&\\&\\
\end{tabular}

\begin{tabular}{p{3cm}rl}
& \multicolumn{1}{l}{\Large{\sffamily{Contact}}} & \\
& & \\
& {Address:}
 & Goethe-Universität\\
& & Institut für Linguistik\\
& & Norbert-Wollheim-Platz 1\\
& & 60629 Frankfurt am Main\\
& & Germany\\
  &  {E-mail:}     & \href{mailto:bergsma@em.uni-frankfurt.de}{bergsma@em.uni-frankfurt.de}\\
  &  {Website:}
	& \href{http://user.uni-frankfurt.de/bergsma/}{user.uni-frankfurt.de/bergsma/} \\&\\&\\&\\
\end{tabular}

\begin{tabular}{p{3cm}p{13cm}}
& \multicolumn{1}{l}{\Large{\sffamily{Education}}} \\
& \\
\sffamily10/2017-09/2020 & PhD \textsc{Linguistics}, \textbf{Goethe-Universität}, Frankfurt\\
& Research Training Group on Nominal Modification\\
& Preliminary dissertation title: \tit{Case competition in headless relatives} \\
\sffamily09/2019, & research stay at \textbf{Masarykova univerzita}, Brno\\
\sffamily02/2019-04/2019 & \\
\sffamily03/2018-05/2018 & research stay at \textbf{University of Pennsylvania}, Philadelphia\\
\sffamily07/08/2018 & EGG summer school 2018, \tbf{University of Banja Luka}\\
\sffamily07/08/2017 & EGG summer school 2017, \tbf{Univerzita Palackého v Olomouci}\\&\\&\\
\sffamily09/2013-08/2015 & Research Master \textsc{Linguistics}, \textbf{Universiteit van Amsterdam}\\
& A research-focused two-year master program (in English)\\
& MA-thesis: \emph{To serve double duty under syncretism - How Nanosyntax and grafting account for the free relative construction}\\&\\&\\
\sffamily09/2010-08/2013 & Bachelor \textsc{Linguistics}, \textbf{Rijksuniversiteit Groningen} - \small\emph{cum laude}\\
& BA-thesis: \emph{Does Knowledge about Phonological Correspondences Contribute to the Intelligibility of a Related Language? A Study with Speakers of Dutch Learning Frisian.}\\
\sffamily09/2010-08/2013& Honours College program (25\% extra credits) \\&\\
\end{tabular}



Organization
Co-organizer, Workshop on Differential internal possessors at Syntax of the World’sLanguages 8, Inalco, 3 September 2018, Paris, France



\section{Publications}
\begin{tabular}{p{3cm}l}
under review & Bergsma, Fenna. daarmee\\


2019 & Bergsma, Fenna. Mismatches in free relatives - grafting nanosyntactic trees. \emph{Glossa: A Journal of General Linguistics 4}(1), 119. DOI: \href{http://doi.org/10.5334/gjgl.821}{http://doi.org/10.5334/gjgl.821}\\&\\
2019 & Bergsma, Fenna. \href{https://repository.upenn.edu/pwpl/vol25/iss1/6}{The Role of Prepositions in Case Mismatches in Free Relatives.} In: A. Creemers and C. Richter.\textit{ University of Pennsylvania Working Papers in Linguistics 25.1.} 41-50. \\&\\
2014 & Bergsma, Fenna, Femke Swarte and Charlotte Gooskens. Does Instruction about Phonological Correspondences Contribute to the Intelligibility of a Related Language? A Study with Speakers of Dutch Learning Frisian. \emph{Dutch Journal of Applied Linguistics, 3}(1), 45-61. DOI: \href{https://doi.org/10.1075/dujal.3.1.03ber}{https://doi.org/10.1075/dujal.3.1.03ber}\\&\\
\end{tabular}

\section{Talks}

\subsection{Invited talks}
\begin{tabular}{p{3cm}l}
2018 & ``\emph{Waarmee} and \emph{met wat} in Dutch free relatives'', \emph{Oberseminar English Linguistics (Syntax-Semantics)}, Georg-August-Universität Göttingen, October 30.\\&\\
2018 & ``Grafting nanosyntactic trees: an analysis of case mismatches in free relatives'', \emph{Kolloquium Topics in syntax and its interfaces}, Universität Leipzig,  June 26.\\&\\
\end{tabular}

\section{Conference talks (peer-reviewed)}
\begin{tabular}{p{3cm}l}
2019 & ``Gender: a Matter of Size – On Individuation, Mass, and Diminutives in Dutch'', \textit{SinFonIJA 12}, Masarykova univerzita, September 12 - 14.\\
& With Jan Don\\&\\
2019 & ``Verum focus in Frisian'', poster at \emph{GLOW 42}, Universitetet i Oslo, May 7-11.\\&\\
2019 & ``PPs and DPs in non-matching free relatives'', \emph{Exploring Nanosyntax, LSA annual meeting}, New York, January 3-6.\\&\\
2018 & ``PPs and DPs in free relatives: \emph{waarmee} and \emph{met wat} in Dutch'', \emph{On the place of case in grammar (PlaCiG)}, Rethymnon, October 18-20.\\&\\
2018 & ``Case mismatches in free relative constructions'', \emph{CGG 28}, Universitat Rovira i Virgili, May 30-June 1.\\&\\
2018 & ``Mismatches in free relatives'', \emph{GLOW 41}, Research Institute for Linguistics of the Hungarian Academy of Sciences, April 10-14.\\&\\
2018 & ``The power of syncretisms: how syncretisms can serve double duty'', \emph{Penn Linguistics Conference 42}, University of Pennsylvania, March 23-25.\\&\\
2018 & ``Syncretism = shared syntax + shared spellout'' \emph{ConSOLE XXVI}, University College London, February 14-16.\\&\\
2016 & ``To serve double duty under syncretism'' \emph{ConSOLE XXIV}, University of York and York St John University, January 6-8.\\
& With Jan Don\\&\\
\end{tabular}

\begin{tabular}{p{3cm}l}
2015 & ``To serve double duty under syncretism'' \emph{Morphologydays 2015}, Katholieke Universiteit Leuven, December 17-18.\\
& With Jan Don\\&\\
2013 & ``Does Knowledge About Linguistic Differences Contribute to Receptive Multilingualism? - A Pilot Study With Speakers of Dutch Learning Frisian'' \emph{ExAPP 2013: Experimental Approaches to Perception and Production of Language Variation}, Københavns Universitet, March 20.\\&\\
\end{tabular}

\section{Other talks (non peer-reviewed)}
\begin{tabular}{p{3cm}l}
2019 & ``Towards deriving a typology of case mismatches in free relatives'', \emph{Colloquium Graduiertenkolleg Nominal Modification}, October 22.\\&\\
2018 & ``Mismatches in free relatives - grafting nanosyntactic trees'', \emph{Nanosyntax Weblab}, online, November 30.\\&\\
2018 & ``On the distribution of \emph{waarmee} and \emph{met wat} in Dutch free relatives'', \emph{Colloquium Graduiertenkolleg Nominal Modification}, Goethe-Universität Frankfurt, October 16.\\&\\
2018 & ``Mismatches in free relatives'', \emph{GK Nominal Modification 2018 Summer Retreat}, Fulda, May 25-26.\\&\\
2017 & ``Syncretism = shared syntax + shared spellout'' \emph{Colloquium Graduiertenkolleg Nominal Modification}, Goethe-Universität Frankfurt, December 5.\\&\\
2015 & ``(De) Mij(n(e(s))) - Complex Possessives in Dutch'' \emph{Taalkunde in Nederland-dag 2015}, Universiteit Utrecht, February 7. Presented by Yvonne van Baal and Fenna Bergsma \\& With Yvonne van Baal and Jan Don \\&\\
2013 & ``How does reality influence performance on false belief tasks?'' \emph{34th TABU Dag 2013}, Rijksuniversiteit Groningen, June 14. Presented by Jidde Jacobi and Maike Tromp \\
& With Jidde Jacobi, Maike Tromp, Maret Hans and Bart Hollebrandse\\&\\
2013 & ``Does Knowledge About Linguistic Differences Contribute to Receptive Multilingualism? - A Pilot Study With Speakers of Dutch Learning Frisian'' \emph{Anéla/Viot Juniorendag 2013}, Rijksuniversiteit Groningen, March 3.\\&\\
2012 & ``Interventiemethoden voor receptieve meertaligheid - Een case study met
Nederlanders die Fries leren verstaan.'' \emph{Dei fan de Fryske Taalkunde 2012}, Fryske Akademy, December 14.\\&\\
\end{tabular}

\section{Teaching Experience}
\begin{tabular}{p{3cm}l}
	\textsc{2019} & Summer Term. DP Morphology, MA Seminar\\
	& Goethe-Universität Frankfurt\\\multicolumn{2}{c}{} \\
	\textsc{2012} & February-April. Teaching assistant SPSS Statistics practical\\
	& University of Groningen\\\multicolumn{2}{c}{} \\
\end{tabular}

\section{Relevant nonacademic work experience}
\begin{tabular}{p{3cm}l}
\textsc{2015 - 2017} & Project manager at \textsc{Stichting Praktijkleren}, Amersfoort\\
\textsc{2006 - 2015} & Developer of teaching materials at \textsc{Stichting Praktijkleren}, Amersfoort\\
\end{tabular}

\section{Languages}
\begin{tabular}{p{3cm}l}
\textsc{Native}: &West Frisian, Dutch\\
\textsc{Fluent}: &English\\
\textsc{Good command}: &German\\&\\&\\
\end{tabular}

\end{document}
